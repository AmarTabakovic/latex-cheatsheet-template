\usepackage[utf8]{inputenc}
\usepackage{amsmath}
\usepackage{amsfonts}
\usepackage{amssymb}
\usepackage{graphicx}
\usepackage[final]{pdfpages}
\usepackage{stmaryrd}
\usepackage{listings}
\usepackage[hidelinks]{hyperref}
\usepackage[english]{babel}
\usepackage{color}
\usepackage{fancyhdr}
\usepackage{xcolor}
\usepackage{url}
\usepackage{sectsty}
\usepackage{etoolbox}
\usepackage{multicol}
\usepackage{wrapfig}
\usepackage{microtype}
\usepackage[top=1mm,bottom=1mm,left=1mm,right=1mm]{geometry}
\usepackage{tikz}
\usepackage[framemethod=tikz]{mdframed}
\usepackage[font=scriptsize]{caption}
\usepackage[compact]{titlesec}
\usepackage{float}

% Graphics and images
\graphicspath{{./images/}}

% Header size
\sectionfont{\fontsize{13}{0}\sffamily}
\subsectionfont{\fontsize{11}{0}\sffamily}
\subsubsectionfont{\fontsize{9}{0}\sffamily}

% Defining colors for syntax highlighting
\definecolor{commentsColor}{rgb}{0.497495, 0.497587, 0.497464}
\definecolor{keywordsColor}{rgb}{0.000000, 0.000000, 0.635294}
\definecolor{stringColor}{rgb}{0.558215, 0.000000, 0.135316}

% Other colors
\definecolor{myblue}{RGB}{15, 30, 86}

% Source: https://denbeke.be/blog/programming/syntax-highlighting-in-latex/
\lstset{
  backgroundcolor=\color{white},
  basicstyle=\ttfamily\scriptsize,
  breakatwhitespace=false,
  breaklines=true,
  commentstyle=\color{commentsColor}\textit,
  deletekeywords={...},
  escapeinside={\%*}{*)},
  extendedchars=true,
  frame=none,
  keepspaces=true,
  keywordstyle=\color{keywordsColor}\bfseries,
  otherkeywords={*,...},
  numbersep=5pt,
  numberstyle=\tiny\color{commentsColor}, 
  rulecolor=\color{black}, 
  showspaces=false,
  showstringspaces=false,
  showtabs=false, 
  stepnumber=1, 
  stringstyle=\color{stringColor},
  tabsize=2, 
  columns=fixed,
  aboveskip=2pt,
  belowskip=2pt
}

% Hyperlinks
\hypersetup{
  colorlinks=true,
  allcolors=black
}

% Source: https://github.com/kenfehling/latex-cheatsheet
\pgfdeclarelayer{background}
\pgfsetlayers{background,main}

\renewcommand{\baselinestretch}{.8}
\pagestyle{empty}

\makeatother
\setlength{\parindent}{0pt}
